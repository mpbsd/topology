\section{Largest and smallest topologies}\label{sec:largest-and-smallest-topologies}

\begin{problem}
  The intersection of any collection of topologies for \(X\) is a topology for
  \(X\).
  \label{problem:the-intersection-of-topologies-is-again-a-topology}
\end{problem}

\begin{solution}
  Given a non empty collection
  \(\left\{\mathcal{T}_{\gamma}:\gamma\in\Gamma\right\}\) of topologies for
  \(X\), the family
  \[
    \mathcal{T}=\bigcap\left\{\mathcal{T}_{\gamma}:\gamma\in\Gamma\right\}
  \]
  of the common sets among the \(\mathcal{T}_{\gamma}\) is also a topology for
  \(X\) because:
  \begin{enumerate}
    \item
      The intersection of any two members of \(\mathcal{T}\) is a member of
      \(\mathcal{T}\):

      If \(U\) and \(V\) both are in \(\mathcal{T}\) then, for each
      \(\gamma\in\Gamma\), they are also in \(\mathcal{T}_{\gamma}\) and,
      because \(\mathcal{T}_{\gamma}\) is a topology, their intersection
      \(U\cap{V}\) is in \(\mathcal{T}_{\gamma}\). Since this is true for every
      \(\gamma\in\Gamma\), the set \(U\cap{V}\) is in \(\mathcal{T}\).

    \item
      The union of an arbitrary family of sets in \(\mathcal{T}\) is a member
      of \(\mathcal{T}\):

      If the members of a family \(\left\{U_{\alpha}:\alpha\in{A}\right\}\) are
      in \(\mathcal{T}\) then, for each \(\gamma\in\Gamma\), they are also in
      \(\mathcal{T}_{\gamma}\) and, because \(\mathcal{T}_{\gamma}\) is a
      topology, their union
      \begin{equation}
        \bigcup\left\{U_{\alpha}:\alpha\in{A}\right\}
        \label{eq:union-00}
      \end{equation}
      is in \(\mathcal{T}_{\gamma}\). Since this is true for every
      \(\gamma\in\Gamma\), the set in \eqref{eq:union-00} is in
      \(\mathcal{T}\).

  \end{enumerate}
  It should be noticed that \(\mathcal{T}\) is contained in each
  \(\mathcal{T}_{\gamma}\).
\end{solution}

\begin{problem}
  The union of two topologies for \(X\) may not be a topology for \(X\) (unless
  \(X\) consists of at most two points).
  \label{problem:the-union-of-topologies-may-not-be-a-topology}
\end{problem}

\begin{solution}
  We divide the argument into two parts.
  \begin{enumerate}
    \item
      If \(X\) has at least three distinct elements, say \(x,y\) and \(z\), then
      \begin{align}
        \label{eq:topology-t0-for-a-set-with-at-least-three-elements}
        \mathcal{T}_{0}&=\left\{\emptyset,\left\{y\right\},\left\{x,y\right\},X\right\}\\
        \label{eq:topology-t1-for-a-set-with-at-least-three-elements}
        \mathcal{T}_{1}&=\left\{\emptyset,\left\{z\right\},\left\{x,z\right\},X\right\}
      \end{align}
      are topologies for \(X\). However, their union
      \begin{equation}
        \mathcal{T}_{0}\cup\mathcal{T}_{1}=\left\{\emptyset,\left\{y\right\},\left\{z\right\},\left\{x,y\right\},\left\{x,z\right\},X\right\}
      \end{equation}
      is not a topology for \(X\) since both \(\left\{y\right\}\) and
      \(\left\{z\right\}\) belong to \(\mathcal{T}_{0}\cup\mathcal{T}_{1}\) but
      their union \(\left\{y,z\right\}\) does not.
    \item
      If \(X\) has at most two distinct elements, say \(x\) and \(y\),
      then
      \begin{align}
        \label{eq:topology-t0-on-a-two-elements-set}
        \mathcal{T}_{0}&=\left\{\emptyset,X\right\}\\
        \label{eq:topology-t1-on-a-two-elements-set}
        \mathcal{T}_{1}&=\left\{\emptyset,\left\{x\right\},X\right\}\\
        \label{eq:topology-t2-on-a-two-elements-set}
        \mathcal{T}_{2}&=\left\{\emptyset,\left\{y\right\},X\right\}\\
        \label{eq:topology-t3-on-a-two-elements-set}
        \mathcal{T}_{3}&=\left\{\emptyset,\left\{x\right\},\left\{y\right\},X\right\}
      \end{align}
      are the only topologies in existence for \(X\). It follows from equations
      \eqref{eq:topology-t0-on-a-two-elements-set} trough
      \eqref{eq:topology-t3-on-a-two-elements-set} that the union of any two of
      them results in a member of the already mentioned list and as such it
      must be a topology for \(X\).
  \end{enumerate}
\end{solution}

\begin{problem}
  For any collection of topologies for \(X\) there is a unique largest topology
  which is smaller than each member of the collection, and a unique smallest
  topology which is greater than each member of the collection.
  \label{problem:smallest-and-largest-topologies}
\end{problem}

\begin{solution}
  Let there be given a family
  \(\left\{\mathcal{T}_{\gamma}:\gamma\in\Gamma\right\}\) of topologies for
  \(X\). Then:
  \begin{enumerate}
    \item
      We have already seen in problem
      \ref{problem:the-intersection-of-topologies-is-again-a-topology} that
      \[
        \mathcal{T}=\bigcap\left\{\mathcal{T}_{\gamma}:\gamma\in\Gamma\right\}
      \]
      is a topology for \(X\) that is contained in every
      \(\mathcal{T}_{\gamma}\). Not only that, \(\mathcal{T}\)
      \begin{center}
        \textit{contains every topology for \(X\) that is contained in all of the \(\mathcal{T}_{\gamma}\)}
      \end{center}
      and is therefore the largest topology with this property.
    \item
      In problem
      \ref{problem:the-union-of-topologies-may-not-be-a-topology}
      we have learnt that the union
      \begin{equation}
        \mathfrak{T}=\bigcup\left\{\mathcal{T}_{\gamma}:\gamma\in\Gamma\right\}
        \label{eq:the-union-of-the-members-of-t-gamma}
      \end{equation}
      might fail to be a topology for \(X\). Nonetheless, the family
      \begin{equation}
        \mathcal{T}=\bigcap\left\{\mathcal{U}:\mathfrak{T}\subset\mathcal{U}\text{ and }\mathcal{U}\text{ is a topology for }X\right\}
      \end{equation}
      never does so (just notice how
      \[
        \left\{\mathcal{U}:\mathfrak{T}\subset\mathcal{U}\text{ and }\mathcal{U}\text{ is a topology for }X\right\}
      \]
      is not empty by recognizing 
      \[
        2^{X}=\left\{A:A\subset{X}\right\}
      \]
      as one of its members). By its very definition, \(\mathcal{T}\)
      \begin{center}
        \textit{is contained in every topology for \(X\) containing all of the \(\mathcal{T}_{\gamma}\)}
      \end{center}
      and is therefore the smallest topology with this property.
  \end{enumerate}
\end{solution}
