\section{Largest and smallest topologies}

\begin{problem}
  The intersection of any collection of topologies for \(X\) is a topology for
  \(X\).
\end{problem}

\begin{solution}
  If \(\mathcal{T}_{\gamma\in\Gamma}\) is an arbitrary family of topologies for
  a given set \(X\), then 
  \[
    \mathcal{T}=\bigcap_{\gamma\in\Gamma}\mathcal{T}_{\gamma},
  \]
  is also a topology for \(X\), for the following reasons:
  \begin{enumerate}
    \item
      the intersection of any two members of \(\mathcal{T}\) is a member of
      \(\mathcal{T}\):

      If \(U\) and \(V\) are in \(\mathcal{T}\) then, for each
      \(\gamma\in\Gamma\), they also are in \(\mathcal{T}_{\gamma}\) and,
      because \(\mathcal{T}_{\gamma}\) is a topology, the set \(U\cap{V}\) is
      in \(\mathcal{T}_{\gamma}\). Since this is true for each
      \(\gamma\in\Gamma\), \(U\cap{V}\) is in \(\mathcal{T}\).

    \item
      the union of an arbitrary family of sets in \(\mathcal{T}\) is a member
      of \(\mathcal{T}\):

      If \(\left\{U_{\alpha}:\alpha\in{A}\right\}\) is a family of sets
      \(U_{\alpha}\in\mathcal{T}\) then, for each \(\gamma\in\Gamma\), it is
      also a family of sets \(U_{\alpha}\in\mathcal{T}_{\gamma}\) and, because
      \(\mathcal{T}_{\gamma}\) is a topology, the set
      \begin{equation}
        \bigcup_{\alpha\in{A}}U_{\alpha}
        \label{eq:union-00}
      \end{equation}
      is in \(\mathcal{T}_{\gamma}\). Since this is true for each
      \(\gamma\in\Gamma\), the set in \eqref{eq:union-00} is in
      \(\mathcal{T}\).

  \end{enumerate}
\end{solution}

\begin{problem}
  The union of two topologies for \(X\) may not be a topology for \(X\) (unless
  \(X\) consists of at most two points).
\end{problem}

\begin{solution}
  adf
\end{solution}

\begin{problem}
  For any collection of topologies for \(X\) there is a unique largest topology
  which is smaller than each member of the collection, and a unique smallest
  topology which is greater than each member of the collection.
\end{problem}
