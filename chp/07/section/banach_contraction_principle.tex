\section{Banach's contraction principle}

This section closely follows~\cite{a_simple_proof_of_the_banach_contraction_principle}.

\begin{definition}\label{def:contracting-mapping}
  Let \(X\) be a metric space. Then, for any constant \(K\in(0,1)\), a mapping
  \(f:X\to{X}\) is said to be a \(K\)-contraction on the space \(X\) in case
  \[
    d(f(x_{1}),f(x_{2}))\leqslant{Kd(x_{1},x_{2})},
  \]
  for all \(x_{1},x_{2}\in{X}\).
\end{definition}

\begin{proposition}[Fundamental inequality]
  Let \(f\) be a \(K\)-contraction on a metric space \(X\). Then, we have that
  \begin{equation}\label{eq:fundamental-contraction-inequality}
    (1-K)d(x_{1},x_{2})\leqslant{d(x_{1},f(x_{2}))+d(x_{2},f(x_{2}))},
  \end{equation}
  for all \(x_{1},x_{2}\in{X}\).
\end{proposition}

\begin{proof}
  By the triangle inequality, we have that
  \[
    d(x_{1},x_{2})\leqslant{d(x_{1},f(x_{1}))+d(f(x_{1}),f(x_{2}))+d(f(x_{2}),x_{2})},
  \]
  from which it follows that
  \[
    (1-K)d(x_{1},x_{2})\leqslant{d(x_{1},f(x_{1}))+d(x_{2},f(x_{2}))},
  \]
  for all \(x_{1},x_{2}\in{X}\), as claimed.
\end{proof}

\begin{corollary}
  A contraction has at most one fixed point.
\end{corollary}

\begin{proof}
  If \(x_{1},x_{2}\in{X}\) are fixed points of a \(K\)-contraction \(f\) on
  \(X\), we get from~\eqref{eq:fundamental-contraction-inequality} that
  \[
    (1-K)d(x_{1},x_{2})
    \leqslant{d(x_{1},f(x_{1}))+d(x_{2},f(x_{2}))}
    \leqslant{d(x_{1},x_{1})+d(x_{2},x_{2})}
    =0,
  \]
  from which it follows that \(d(x_{1},x_{2})=0\), since \(0<K<1\).
\end{proof}

\begin{proposition}\label{proposition:the-iterates-of-any-point-under-a-contraction-is-always-a-cauchy-sequence}
  Let \(f\) be a \(K\)-contraction on a metric space \(X\). Then, for any point
  \(x\in{X}\), the sequence
  \[
    x_{1}=x,\quad{x_{n+1}=f(x_{n})},
  \]
  is a Cauchy sequence in \(X\).
\end{proposition}

\begin{proof}
  Let \(x\in{X}\) be given. Then, we get that
  \[
    \forall{m}\in\naturals:
    \quad{x_{m+1}=f^{m}(x)},
  \]
  as well as
  \[
    \forall{m}\in\naturals:
    \quad{d(f^{m}(x),f^{m+n}(x))\leqslant{K^{m}}d(x,f^{n}(x))}
    \quad{(\forall{n}\in\naturals)},
  \]
  by induction on \(m\). It follows that
  \begin{equation}
    \begin{split}
      (1-K)d(x_{m},x_{n})
      &=
      (1-K)d(f^{m-1}(x),f^{n-1}(x))
      \\
      &\leqslant{d(f^{m-1}(x),f^{m}(x))+d(f^{n-1}(x),f^{n}(x))}
      \\
      &\leqslant{\left(K^{m-1}+K^{n-1}\right)d(x,f(x))},
    \end{split}
  \end{equation}
  for all \(m,n\in\naturals\), once again
  by~\eqref{eq:fundamental-contraction-inequality}. Thus,
  \(d(x_{m},x_{n})\to{0}\) as \(m,n\to\infty\).
\end{proof}

\begin{corollary}\label{corollary:banachs-contraction-principle}
  Let \(X\) be a complete metric space. Then, a \(K\)-contraction \(f\) on
  \(X\) has exactly one fixed point, say \(y\in{X}\). Moreover, if \(x\) is any
  point in \(X\), we further have \(y=\lim\limits_{m\to\infty}f^{m}(x)\).
\end{corollary}

\begin{proof}
  By
  Proposition~\ref{proposition:the-iterates-of-any-point-under-a-contraction-is-always-a-cauchy-sequence},
  the sequence
  \[
    x_{m+1}=f^{m}(x),
  \]
  is, for any \(x\in{X}\), a Cauchy sequence of points in \(X\). Since \(X\) is
  now a complete metric space, there does exist a \(y\in{X}\) with which to
  write \(y=\lim\limits_{m\to\infty}f^{m}(x)\). Then, we have that
  \[
    f(y)
    =
    f\left(\lim\limits_{m\to\infty}f^{m}(x)\right)
    =
    \lim\limits_{m\to\infty}f^{m+1}(x)
    =
    y.
  \]
  \[
    d(x_{m},y)\leqslant{\frac{K^{m-1}d(x,f(x))}{1-K}}
  \]
\end{proof}
