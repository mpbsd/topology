\section{Banach's contraction principle}

This section closely follows~\cite{a_simple_proof_of_the_banach_contraction_principle}.

\begin{definition}\label{def:contracting-mapping}
  Let \(X\) be a metric space. Then, a contraction on \(X\) is a mapping
  \(f:X\to{X}\) such that there exists a real constant \(0<K<1\) with
  \[
    d(f(x_{1}),f(x_{2}))\leqslant{Kd(x_{1},x_{2})},
  \]
  for all \(x_{1},x_{2}\in{X}\). The constant \(K\) is a contracting constant
  for \(f\).
\end{definition}

\begin{proposition}[Fundamental inequality]
  Let \(f:X\to{X}\) be a contraction with contracting constant \(K\). Then, we
  have that
  \begin{equation}\label{eq:fundamental-contraction-inequality}
    (1-K)d(x_{1},x_{2})\leqslant{d(x_{1},f(x_{2}))+d(x_{2},f(x_{2}))},
  \end{equation}
  for all \(x_{1},x_{2}\in{X}\).
\end{proposition}

\begin{proof}
  By the triangle inequality, we have that
  \[
    d(x_{1},x_{2})\leqslant{d(x_{1},f(x_{1}))+d(f(x_{1}),f(x_{2}))+d(f(x_{2}),x_{2})},
  \]
  from which it then follows that
  \[
    (1-K)d(x_{1},x_{2})\leqslant{d(x_{1},f(x_{1}))+d(x_{2},f(x_{2}))},
  \]
  for all \(x_{1},x_{2}\in{X}\), as we claimed.
\end{proof}

\begin{corollary}
  A contraction has at most one fixed point.
\end{corollary}

\begin{proof}
  It follows from~\eqref{eq:fundamental-contraction-inequality} that if
  \(x_{1},x_{2}\in{X}\) are fixed points of a contraction \(f:X\to{X}\) with
  constant \(K\), we then get that
  \[
    (1-K)d(x_{1},x_{2})
    \leqslant{d(x_{1},f(x_{1}))+d(x_{2},f(x_{2}))}
    \leqslant{d(x_{1},x_{1})+d(x_{2},x_{2})}
    =0,
  \]
  and because \(0<K<1\), we further get \(d(x_{1},x_{2})=0\).
\end{proof}

\begin{proposition}\label{proposition:the-iterates-of-any-point-under-a-contraction-is-always-a-cauchy-sequence}
  Let \(f:X\to{X}\) be a contraction with constant \(K\). Then, the sequence of
  iterates under \(f\) of any point \(x\) in \(X\), that is, the sequence
  \[
    x_{1}=x,\quad{x_{n+1}=f(x_{n})},
  \]
  is always a Cauchy sequence in \(X\).
\end{proposition}

\begin{proof}
  Let \(x\in{X}\) be given. Then, we get that
  \[
    \forall{m}\in\naturals:
    \quad{x_{m+1}=f^{m}(x)},
  \]
  as well as
  \[
    \forall{m}\in\naturals:
    \quad{d(f^{m}(x),f^{m+n}(x))\leqslant{K^{m}}d(x,f^{n}(x))}
    \quad{(\forall{n}\in\naturals)},
  \]
  by induction on \(m\). It follows that
  \begin{equation}
    \begin{split}
      (1-K)d(x_{m},x_{n})
      &=
      (1-K)d(f^{m-1}(x),f^{n-1}(x))
      \\
      &\leqslant{d(f^{m-1}(x),f^{m}(x))+d(f^{n-1}(x),f^{n}(x))}
      \\
      &\leqslant{\left(K^{m-1}+K^{n-1}\right)d(x,f(x))},
    \end{split}
  \end{equation}
  for all \(m,n\in\naturals\), once again
  by~\eqref{eq:fundamental-contraction-inequality}. Thus,
  \(d(x_{m},x_{n})\to{0}\) as \(m,n\to\infty\).
\end{proof}

\begin{corollary}\label{corollary:banachs-contraction-principle}
  Let \(X\) be a complete metric space. Then, any contraction \(f:X\to{X}\) has
  exactly one fixed point, say \(y\in{X}\). Moreover, if \(x\in{X}\) is any
  point, then we have \(y=\lim\limits_{m\to\infty}f^{m}(x)\).
\end{corollary}

\begin{proof}
  The iterates \(x_{m+1}=f^{m}(x)\) of any point \(x\in{X}\) through \(f\) form
  a Cauchy sequence in the complete metric space \(X\), by
  Proposition~\ref{proposition:the-iterates-of-any-point-under-a-contraction-is-always-a-cauchy-sequence}.
  So, there does exist a \(y\in{X}\) with which to write
  \(y=\lim\limits_{m\to\infty}f^{m}(x)\). Finally, notice that
  \[
    f(y)=f\left(\lim\limits_{m\to\infty}f^{m}(x)\right)=\lim\limits_{m\to\infty}f^{m+1}(x)=y.
  \]
\end{proof}
