\section{Linear Spaces}\label{sec:linear-spaces}

\subsection{Norm}\label{subsec:norms}

\begin{definition}\label{def:normed-linear-spaces}
  Let \(V\) be a complex vector space. Then, a norm on \(V\) is a function
  \(\norm:V\to\reals\) such that:
  \begin{enumerate}
    \item
      \(\norm[z]\geqslant{0}\) for all \(z\in{V}\);
    \item
      \(\norm[z]=0\iff{z=0}\);
    \item
      \(\norm[\lambda{z}]=\abs{\lambda}\norm[z]\) for all
      \((\lambda,z)\in\complexfield\times{V}\);
    \item
      \(\norm[z+w]\leqslant{\norm[z]+\norm[w]}\) for all
      \((z,w)\in{V\times{V}}\).
  \end{enumerate}
  A normed space is a pair \((V,\norm)\) where \(V\) is vector space and
  \(\norm\) is a norm on \(V\).
\end{definition}

\subsection{The \(p\)-norm}\label{subsec:p-norm}

In this section we closely follow the lines
of~\cite{yet_another_proof_of_minkowskis_inequality}.

\begin{definition}
  Let \(p\geqslant{1}\) be given. Then, the \(p\)-norm on \(\complexfield^{n}\)
  is given by
  \[
    \norm<n>(p)[z]
    =
    \sqrt[p]{\abs{z_{1}}^{p}+\cdots+\abs{z_{n}}^{p}},
  \]
  for all \(z=(z_{1},\ldots,z_{n})\in\complexfield^{n}\).
\end{definition}

\begin{lemma}\label{lemma:on-norms-and-the-convexity-of-the-unitary-closed-ball}
  Let \(V\) be a complex vector space and \(N:V\to\reals\) be a real-valued
  function satisfying:
  \begin{enumerate}
    \item
      \(\norm[z]\geqslant{0}\) for all \(z\in{V}\);
    \item
      \(\norm[z]=0\iff{z=0}\);
    \item
      \(\norm[\lambda{z}]=\abs{\lambda}\norm[z]\) for all
      \((\lambda,z)\in\complexfield\times{V}\).
  \end{enumerate}
  Then, \(\norm\) is a norm on \(V\) whenever the unit ball
  \(\left\{z\in{V}:\norm[z]\leqslant{1}\right\}\) is a convex set.
\end{lemma}

\begin{proof}
  Let \(z\) and \(w\) be nonzero vectors in \(V\). Then, \(z/\norm[z]\) and
  \(w/\norm[w]\) are unit vectors and their convex combination
  \[
    \frac{z+w}{\norm[z]+\norm[w]}
    =
    \frac{\norm[z]}{\norm[z]+\norm[w]}\frac{z}{\norm[z]}
    +
    \frac{\norm[w]}{\norm[z]+\norm[w]}\frac{w}{\norm[w]}
    ,
  \]
  lies in the unit ball. Therefore, we get that
  \[
    \frac{\norm[z+w]}{\norm[z]+\norm[w]}
    =
    N\left(\frac{z+w}{\norm[z]+\norm[w]}\right)\leqslant{1},
  \]
  from which it then follows that the triangle inequality
  \[
    \norm[z+w]\leqslant{\norm[z]+\norm[w]},
  \]
  holds.
\end{proof}

\begin{lemma}\label{lemma:concavity-of-compositions-of-concave-functions}
  Let \(S\) be a convex subset of a vector space, \(J\) an interval of the real
  line and \(g:S\to{J}\) a concave function. Then, the composition
  \(f\circ{g}:S\to\reals\) is also a concave function whenever \(f:J\to\reals\)
  is concave and increasing.
\end{lemma}

\begin{proof}
  \begin{align*}
    t(f\circ{g})(z)+(1-t)(f\circ{g})(w)
    &\leqslant{f(tg(z)+(1-t)g(w))}\\
    &\leqslant{(f\circ{g})(tz+(1-t)w)},
  \end{align*}
  for all \((t,z,w)\in{[0,1]\times{S}\times{S}}\), where the first inequality
  follows from the concavity of \(f\) and the second one follows from the
  concavity of \(g\) and the fact that \(f\) increases.
\end{proof}

\begin{lemma}\label{lemma:concavity-of-sums-of-concave-functions}
  Let
  \[
    \left(g_{i}:V_{i}\to\reals\right)_{i=1}^{n},
  \]
  be concave functions and \(S\subset{V_{1}\times\cdots\times{V_{n}}}\) be a
  convex set. Then, for any constant \(c\in\reals\), the function
  \[
    g:S\to\reals,\quad{(z_{1},\ldots,z_{n})\mapsto{c+\sum_{i=1}^{n}g_{i}(z_{i})}},
  \]
  is concave.
\end{lemma}

\begin{proof}
  As a matter of fact, we have that
  \begin{align*}
    t(g-c)(z)+(1-t)(g-c)(w)
    &=
    \sum_{i=1}^{n}\left(tg_{i}(z_{i})+(1-t)g_{i}(w_{i})\right)
    \\
    &\leqslant
    \sum_{i=1}^{n}g_{i}\left(tz_{i}+(1-t)w_{i}\right)
    \\
    &=
    (g-c)\left(tz+(1-t)w\right),
  \end{align*}
  for all \((t,z,w)\in{[0,1]\times{S}\times{S}}\), where
  \(z=(z_{1},\ldots,z_{n})\) and \(w=(w_{1},\ldots,w_{n})\).
\end{proof}

\begin{lemma}\label{lemma:two-concave-functions}
  The functions
  \begin{equation}\label{eq:f-function}
    f:[0,1]\to\reals,\quad{x\mapsto{\sqrt[p]{x}}},
  \end{equation}
  and
  \begin{equation}\label{eq:k-function}
    k:[0,\infty)\to\reals,\quad{x\mapsto{-x^{p}}},
  \end{equation}
  are both concave, \(f\) increases and \(k\) decreases.
\end{lemma}

\begin{proof}
  An exercise.
\end{proof}

\begin{proposition}\label{proposition:the-p-norm-is-in-fact-a-norm}
  \(\norm<n>(p)\) is a norm on \(\complexfield^{n}\), for every
  \(p\geqslant{1}\).
\end{proposition}

\begin{proof}
  By Lemma~\ref{lemma:on-norms-and-the-convexity-of-the-unitary-closed-ball},
  it suffices to show that the unit ball
  \[
    B^{n}_{p}
    =
    \left\{z\in\complexfield^{n}:\norm<n>(p)[z]\leqslant{1}\right\},
  \]
  is a convex subset of \(\complexfield^{n}\). We proceed by induction on
  \(n\). It's clear that the closed disk
  \(B^{1}_{p}=\left\{z\in\complexfield:\abs{z}\leqslant{1}\right\}\) is a
  convex subset of the complex plane. Next, suppose that the ball
  \[
    B^{n}_{p}
    =
    \left\{z\in\complexfield^{n}:\sum_{i=1}^{n}\abs{z_{i}}^{p}\leqslant{1}\right\},
  \]
  is a convex subset of \(\complexfield^{n}\). To show that
  \[
    B^{n+1}_{p}
    =
    \left\{z\in\complexfield^{n+1}:\sum_{i=1}^{n+1}\abs{z_{i}}^{p}\leqslant{1}\right\},
  \]
  is a convex subset of \(\complexfield^{n+1}\), we introduce the function
  \begin{equation}\label{eq:h-function}
    h:B^{n}_{p}\to{[0,1]},
    \quad{(z_{1},\ldots,z_{n})\mapsto\left(1-\sum_{i=1}^{n}\abs{z_{i}}^{p}\right)^{\frac{1}{p}}}.
  \end{equation}
  Our goal is to show that \(h\) is a concave function, because it will then follow that
  the set
  \[
    B^{n+1}_{p}
    =
    \left\{(z,\mu)\in{B^{n}\times\complexfield}:\abs{\mu}\leqslant{h(z)}\right\},
  \]
  is convex. To see this, suppose that
  \((t,(z,\mu),(w,\nu))\in{[0,1]\times{B^{n+1}_{p}}\times{B^{n+1}_{p}}}\) and
  observe that
  \[
    \abs{t\mu+(1-t)\nu}
    \leqslant{t\abs{\mu}+(1-t)\abs{\nu}}
    \leqslant{th(z)+(1-t)h(w)}
    \leqslant{h(tz+(1-t)w}),
  \]
  so that \(t(z,\mu)+(1-t)(w,\nu)\in{B^{n+1}_{p}}\). In order to accomplish
  that, we may realize the function~\eqref{eq:h-function} as the composite
  \(h=f\circ{g}\), where
  \begin{equation}\label{eq:g-function}
    g:B^{n}_{p}\to[0,1],
    \quad{(z_{1},\ldots,z_{n})\mapsto{1-\sum_{i=1}^{n}\abs{z_{i}}^{p}}},
  \end{equation}
  and
  \begin{equation}\label{eq:f-function}
    f:[0,1]\to\reals,\quad{x\mapsto{\sqrt[p]{x}}}.
  \end{equation}
  By Lemma~\ref{lemma:two-concave-functions}, \(f\) is concave and increases.
  Thus, by Lemma~\ref{lemma:concavity-of-compositions-of-concave-functions},
  \(h=f\circ{g}\) will be seen to be concave as soon as we show that \(g\) is
  also concave. Finally, to see that \(g\) is concave, it suffices to show that
  each
  \[
    g_{i}:\complexfield\to\reals,\quad{z\mapsto{k(\abs{z})=-\abs{z}^{p}}},
  \]
  is a concave function, by
  Lemma~\ref{lemma:concavity-of-sums-of-concave-functions}. Well, we have that
  \begin{align*}
    tg_{i}(z)+(1-t)g_{i}(w)
    &=
    tk\left(\abs{z}\right)+(1-t)k\left(\abs{w}\right)
    \\
    &\leqslant
    k\left(t\abs{z}+(1-t)\abs{w}\right)
    \\
    &\leqslant
    k\left(\abs{tz+(1-t)w}\right)
    =
    g_{i}\left(tz+(1-t)w\right),
  \end{align*}
  for all \((t,z,w)\in{[0,1]\times\complexfield\times\complexfield}\), since,
  by Lemma~\ref{lemma:two-concave-functions}, \(k\) is concave and decreases.
\end{proof}

\subsection{The \(\infty\)-norm}\label{subsec:infty-norm}

\begin{definition}
  The \(\infty\)-norm on \(\complexfield^{n}\) is the function given by
  \[
    \norm<n>(\infty)[z]
    =
    \max\limits_{1\leqslant{i}\leqslant{n}}\abs{z_{i}},
  \]
  for all \(z=(z_{1},\ldots,z_{n})\in\complexfield^{n}\).
\end{definition}

\begin{proposition}
  \(\norm<n>(p)\) and \(\norm<n>(\infty)\) are, for each \(p\geqslant{1}\),
  equivalent norms on \(\complexfield^{n}\).
\end{proposition}

\begin{proof}
  Observe that
  \[
    \max\limits_{1\leqslant{i}\leqslant{n}}\abs{z_{i}}
    \leqslant\sqrt[p]{\abs{z_{1}}^{p}+\cdots+\abs{z_{n}}^{p}}
    \leqslant\sqrt[p]{n}\max\limits_{1\leqslant{i}\leqslant{n}}\abs{z_{i}},
  \]
  that is, that
  \begin{equation}\label{eq:equivalence-of-p-and-infty-norms}
    \norm<n>(\infty)[z]
    \leqslant
    \norm<n>(p)[z]
    \leqslant
    \sqrt[p]{n}\norm<n>(\infty)[z],
  \end{equation}
  for all \(z=(z_{1},\ldots,z_{n})\in\complexfield^{n}\) and \(p\geqslant{1}\),
  which proves the proposition. At this point though, we think it's worth
  noticing that it follows from
  equation~\eqref{eq:equivalence-of-p-and-infty-norms} above that
  \[
    \lim\limits_{p\to\infty}\norm<n>(p)[z]=\norm<n>(\infty)[z],
  \]
  for each \(z\in\complexfield^{n}\), since
  \(\lim\limits_{p\to\infty}\sqrt[p]{n}=1\).
\end{proof}

