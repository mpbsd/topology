\section{Metric Spaces}\label{sec:metric-spaces}

\begin{definition}
  A metric on a set \(X\) is a function \(d:X\times{X}\to\reals\) such that:
  \begin{align}
      & d(x,y)=d(y,x),                  \label{axiom:metric-space-01} \\
      & d(x,z)\leqslant{d(x,y)+d(y,z)}, \label{axiom:metric-space-02} \\
      & x=y\implies{d(x,y)=0}, and      \label{axiom:metric-space-03} \\
      & d(x,y)=0\implies{x=y},          \label{axiom:metric-space-04}
  \end{align}
  for all points \(x,y\) and \(z\) in \(X\). A metric space is a pair \((X,d)\)
  where \(d\) is a metric on \(X\).
\end{definition}

We may omit the metric in \((X,d)\) and simply write `\(X\) is a metric space'
if no confusion seems possible as to which metric is being taken on top of
\(X\). Notice that
\[
  d(x,y)
  =
  \frac{2d(x,y)}{2}
  =
  \frac{d(x,y)+d(y,x)}{2}
  \geqslant
  \frac{d(x,x)}{2}
  =
  0,
\]
for all \(x,y\) in \(X\), so that \(d\) is actually a map from \(X\times{X}\)
into the set
\[
  \nonnegativereals
  =
  \left\{x\in\reals:x\geqslant{0}\right\},
\]
of nonnegative real numbers.

\subsection{Examples}

\begin{example}\label{example:the-0-1-metric}
  Let \(X\) be a non-empty set. Then, the function
  \begin{equation*}
    X\times{X}\to\reals,\quad{(x,y)\mapsto{d(x,y)}}=
    \left\{
      \begin{array}{lll}
        1, & \text{if} & x=y, \\
        0, & \text{if} & x\neq{y},
      \end{array}
    \right.
  \end{equation*}
  is a metric on \(X\). Although it's not a very interesting metric, it appears
  frequently enough in counter-examples as to deserve its place in here.
\end{example}

\begin{example}\label{example:subspaces}
  Let \(X\) be a metric space. Then, any non-empty set \(Y\subset{X}\) can be
  made a metric space if one declares the distance \(d(x,y)\) between elements
  \(x,y\in{Y}\) to be the very distance from \(x\) to \(y\) as elements of
  \(X\). In this case, \(Y\) is said to be a subspace of \(X\).
\end{example}

\begin{example}\label{example:the-real-line}
  This is the most important metric space of all.
\end{example}

\begin{example}[Normed linear spaces]\label{example:normed-linear-spaces}
  Let \(\field{F}\) be one of the following number fields, either
  \(\field{F}=\reals\) or \(\field{F}=\complexfield\), and let \(E\) be a
  vector space over \(\field{F}\). Then, a norm on \(E\) is a real valued
  function
  \[
    \norm:E\to\reals,\quad{x\mapsto\norm(x)},
  \]
  such that
  \begin{align}
    &\norm(x)=0\implies{x=0},               \\
    &\norm(\alpha{x})=\abs{\alpha}\norm(x), \\
    &\norm(x+y)\leqslant{\norm(x)+\norm(y)},
  \end{align}
  for all \((\alpha,x,y)\in\reals\times{E}\times{E}\). A normed linear space is
  a pair \((E,\norm)\) where \(E\) is a vector space (over \(\field{F}\)) and
  \(\norm\) is a norm on \(E\). A normed linear space \((E,\norm)\) is always a
  metric space with respect to the metric (prove it!)
  \[
    E\times{E}\to\reals,\quad{(x,y)\mapsto{d(x,y)=\norm(x-y)}}.
  \]
  Notice that \(\norm(x)=d(x,0)\) for all \(x\in{E}\).
\end{example}

\begin{example}[Inner product spaces]
  Let \(E\) be a real vector space. An inner product on \(E\) is a function
  \[
    E\times{E}\to\reals,\quad(x,y)\mapsto\scalarprod{x}{y},
  \]
  such that
  \begin{align}
    \scalarprod{(x+\alpha{y})}{z} & = \scalarprod{x}{z}+\alpha\scalarprod{y}{z},                   \label{axiom:inner-product-01} \\
    \scalarprod{x}{y}             & = \scalarprod{y}{x},                                           \label{axiom:inner-product-02} \\
    \scalarprod{x}{x}             &   \geqslant{0}\quad\text{and}\quad\scalarprod{x}{x}=0\iff{x=0},\label{axiom:inner-product-03}
  \end{align}
  for all \((\alpha,x,y,z)\in\reals\times{E}\times{E}\times{E}\).
  \[
    \norm:E\to\reals,\quad{x\mapsto{\sqrt{\scalarprod{x}{x}}}}.
  \]
  \begin{equation}
    z=x-\frac{\scalarprod{x}{y}}{\norm(y)^{2}}y\in{E}
  \end{equation}
  \begin{align*}
    0\leqslant\scalarprod{z}{z} & =\scalarprod{\left(x-\frac{\scalarprod{x}{y}}{\norm(y)^{2}}y\right)}{\left(x-\frac{\scalarprod{x}{y}}{\norm(y)^{2}}y\right)}               \\
                                & =\norm(x)^{2}-2\frac{\left(\scalarprod{x}{y}\right)^{2}}{\norm(y)^{2}}+\left(\frac{\scalarprod{x}{y}}{\norm(y)^{2}}\right)^{2}\norm(y)^{2} \\
                                & =\norm(x)^{2}-\frac{(\scalarprod{x}{y})^{2}}{\norm(y)^{2}}
  \end{align*}
  from what it follows that
  \(\abs{\scalarprod{x}{y}}\leqslant\norm(x)\norm(y)\) for all
  \((x,y)\in{E\times{(E\setminus\left\{0\right\})}}\).
  \begin{align*}
    \norm(x+y)^{2} & =\scalarprod{(x+y)}{(x+y)}                                   \\
                   & =\scalarprod{x}{x}+2\scalarprod{x}{y}+\scalarprod{y}{y}      \\
                   & \leqslant{\norm(x)^{2}+\abs{\scalarprod{x}{y}}+\norm(y)^{2}} \\
                   & \leqslant{\norm(x)^{2}+2\norm(x)\norm(y)+\norm(y)^{2}}       \\
                   & \leqslant\left(\norm(x)+\norm(y)\right)^{2},
  \end{align*}
  from what it follows that \(\norm(x+y)\leqslant{\norm(x)+\norm(y)}\) for all
  \((x,y)\in{E\times{E}}\).
\end{example}

