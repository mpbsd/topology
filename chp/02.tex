\chapter{Metric Spaces}\label{chp:metric-spaces}

\section{Normed linear spaces}

In this section we closely follow the lines
of~\cite{yet_another_proof_of_minkowskis_inequality}.

\begin{definition}\label{def:normed-linear-spaces}
  Let \(V\) be a complex vector space. Then, a norm on \(V\) is real-valued
  function \(N:V\to\reals\) such that:
  \begin{enumerate}
    \item
      \(N(v)\geqslant{0}\) for all \(v\in{V}\);
    \item
      \(N(v)=0\iff{v=0}\);
    \item
      \(N(\lambda{v})=\abs{\lambda}N(v)\) for all
      \((\lambda,v)\in\complexfield\times{V}\);
    \item
      \(N(v+w)\leqslant{N(v)+N(w)}\) for all \((v,w)\in{V\times{V}}\).
  \end{enumerate}
\end{definition}

\begin{definition}[\(p\)-norm]
  Let \(p\leqslant{1}\). Then, the \(p\)-norm on \(\complexfield^{n}\) is given by
  \[
    \norm<n>(p)[z]
    =
    \left(\abs{z_{1}}^{p}+\cdots+\abs{z_{n}}^{p}\right)^{\frac{1}{p}},
  \]
  for all \(z=(z_{1},\ldots,z_{n})\in\complexfield^{n}\).
\end{definition}

\begin{lemma}
  Let \(V\) be a complex vector space and \(N:V\to\reals\) be a real-valued
  function satisfying:
  \begin{enumerate}
    \item
      \(N(v)\geqslant{0}\) for all \(v\in{V}\);
    \item
      \(N(v)=0\iff{v=0}\);
    \item
      \(N(\lambda{v})=\abs{\lambda}N(v)\) for all
      \((\lambda,v)\in\complexfield\times{V}\).
  \end{enumerate}
  Then, \(N\) is a norm on \(V\) whenever the unit ball
  \(\left\{v\in{V}:N(v)\leqslant{1}\right\}\) is a convex set.
\end{lemma}

\begin{proof}
  Let \(v\) and \(w\) be nonzero vectors in \(V\). Then, \(v/N(v)\) and
  \(w/N(w)\) are unit vectors and as such their convex combination
  \[
    \frac{v+w}{N(v)+N(w)}
    =
    \frac{N(v)}{N(v)+N(w)}\frac{v}{N(v)}
    +
    \frac{N(w)}{N(v)+N(w)}\frac{w}{N(w)}
    ,
  \]
  lies in the unit ball. Therefore, we get that
  \[
    \frac{N(v+w)}{N(v)+N(w)}
    =
    N\left(\frac{v+w}{N(v)+N(w)}\right)\leqslant{1},
  \]
  from which it then follows that the triangle inequality
  \[
    N(v+w)\leqslant{N(v)+N(w)},
  \]
  holds.
\end{proof}

To establish Minkowski's inequality, it thus suffices to show that the unit
ball
\(B^{n}_{p}=\left\{z\in\complexfield^{n}:\norm<n>(p)[z]\leqslant{1}\right\}\),
is a convex subset of \(\complexfield^{n}\). We proceed by induction on \(n\).
It's clear that the closed disk
\(B^{1}_{p}=\left\{z\in\complexfield:\abs{z}\leqslant{1}\right\}\) is a convex
subset of the complex plane. Next, suppose that the ball
\[
  B^{n}_{p}
  =
  \left\{z\in\complexfield^{n}:\abs{z_{1}}^{p}+\cdots+\abs{z_{n}}^{p}\leqslant{1}\right\},
\]
is a convex subset of \(\complexfield^{n}\). To show that
\[
  B^{n+1}_{p}
  =
  \left\{z\in\complexfield^{n+1}:\abs{z_{1}}^{p}+\cdots+\abs{z_{n}}^{p}+\abs{z_{n+1}}^{p}\leqslant{1}\right\},
\]
is a convex subset of \(\complexfield^{n+1}\), we introduce the function
\(h:B^{n}_{p}\to[0,1]\) given by
\[
	h(z)=\left(1-\abs{z_{1}}^{p}-\cdots-\abs{z_{n}}^{p}\right)^{\frac{1}{p}},
\]
for all \(z=(z_{1},\ldots,z_{n})\in{B^{n}_{p}}\). Our goal is to show that \(h\) is a concave function, that is, that
\[
	th(z)+(1-t)h(w)\leqslant{h(tz+(1-t)w)},
\]
for all \((t,z,w)\in{[0,1]\times{B^{n}_{p}}\times{B^{n}_{p}}}\), for once we
are able to establish that \(h\) satisfies this inequality on the convex set
\(B^{n}_{p}\), it will follow that
\[
	B^{n+1}_{p}
  =
  \left\{(z,\mu)\in{B^{n}\times\complexfield}:\abs{\mu}\leqslant{h(z)}\right\},
\]
is convex. To see this, suppose that
\((t,(z,\mu),(w,\nu))\in{[0,1]\times{B^{n+1}_{p}}\times{B^{n+1}_{p}}}\) and
observe that
\[
	\abs{t\mu+(1-t)\nu}
  \leqslant{t\abs{\mu}+(1-t)\abs{\nu}}
  \leqslant{th(z)+(1-t)h(w)}
  \leqslant{h(tz+(1-t)w}),
\]
so that \(t(z,\mu)+(1-t)(w,\nu)\in{B^{n+1}_{p}}\).

\section{Metric Spaces}\label{sec:metric-spaces}

\begin{definition}
  A metric on a set \(X\) is a function \(d:X\times{X}\to\reals\) such that:
  \begin{align}
      & d(x,y)=d(y,x),                  \label{axiom:metric-space-01} \\
      & d(x,z)\leqslant{d(x,y)+d(y,z)}, \label{axiom:metric-space-02} \\
      & x=y\implies{d(x,y)=0}, and      \label{axiom:metric-space-03} \\
      & d(x,y)=0\implies{x=y},          \label{axiom:metric-space-04}
  \end{align}
  for all points \(x,y\) and \(z\) in \(X\). The pair \((X,d)\) is a metric
  space.
\end{definition}

We may omit the metric in \((X,d)\) and thus write `\(X\) is a metric space' if
no confusion seems possible as to which metric is being taken on \(X\). Notice
that
\[
  d(x,y)
  =
  \frac{2d(x,y)}{2}
  =
  \frac{d(x,y)+d(y,x)}{2}
  \geqslant
  \frac{d(x,x)}{2}
  =
  0,
\]
for all \(x,y\) in \(X\), so that \(d\) is actually a map from \(X\times{X}\)
into the set
\[
  \nonnegativereals
  =
  \left\{x\in\reals:x\geqslant{0}\right\},
\]
of nonnegative reals.

\subsection{Examples}

\begin{example}[The zero-one metric space]\label{example:the-0-1-metric}
  Let \(X\) be a non-empty set. Then, \((X,d)\) is a metric space, where:
  \begin{equation*}
    d(x,y)=
    \left\{
      \begin{array}{lll}
        1 & \text{if} & x=y, \\
        0 & \text{if} & x\neq{y}.
      \end{array}
    \right.
  \end{equation*}
  Although it's not a very interesting metric space, it appears frequently
  enough in counter-examples as to deserve its place in here.
\end{example}

\begin{example}[Subspaces]\label{example:subspaces}
  Let \(X\) be a metric space. Then, a non-empty subset \(Y\) of \(X\) can be
  made a metric space if one declares the distance \(d(x,y)\) between elements
  \(x,y\in{Y}\) to be the very distance from \(x\) to \(y\) as elements of
  \(X\). In this case, \(Y\) is said to be a subspace of \(X\).
\end{example}

\begin{example}[The real line]\label{example:the-real-line}
  This is the most important metric space of all.
\end{example}


\subsection{Balls and spheres}

\subsection{Bounded sets}

\subsection{Distance from sets to sets}

\subsection{Isometries}

\subsection{Pseudo metrics}

\section{Exercises}

\begin{exercise}
  \TeX{}nicians will know.
\end{exercise}
