\chapter{Metric Spaces}\label{chp:metric-spaces}

\section{Metric Spaces}\label{sec:metric-spaces}

\begin{definition}
  A metric on a set \(X\) is a function \(d:X\times{X}\to\reals\) such that:
  \begin{align}
      & d(x,y)=d(y,x),                  \label{axiom:metric-space-01} \\
      & d(x,z)\leqslant{d(x,y)+d(y,z)}, \label{axiom:metric-space-02} \\
      & x=y\implies{d(x,y)=0}, and      \label{axiom:metric-space-03} \\
      & d(x,y)=0\implies{x=y},          \label{axiom:metric-space-04}
  \end{align}
  for all points \(x,y\) and \(z\) in \(X\). The pair \((X,d)\) is a metric
  space.
\end{definition}

When no confusion seems possible, we may ommit the metric in \((X,d)\) and
write simply `\(X\) is a metric space'.

\[
  d(x,y)
  =
  \frac{2d(x,y)}{2}
  =
  \frac{d(x,y)+d(y,x)}{2}
  \geqslant
  \frac{d(x,x)}{2}
  =
  0,
\]

\subsection{Examples}

\subsection{Balls and spheres}

\subsection{Bounded sets}

\subsection{Distance from sets to sets}

\subsection{Isometries}

\subsection{Pseudo metrics}

\section{Exercises}

\begin{exercise}
  \TeX{}nicians will know.
\end{exercise}
