\chapter{Metric Spaces}\label{chp:metric-spaces}

\section{Metric Spaces}\label{sec:metric-spaces}

\begin{definition}
  A metric on a set \(X\) is a function \(d:X\times{X}\to\reals\) such that:
  \begin{align}
      & d(x,y)=d(y,x),                  \label{axiom:metric-space-01} \\
      & d(x,z)\leqslant{d(x,y)+d(y,z)}, \label{axiom:metric-space-02} \\
      & x=y\implies{d(x,y)=0}, and      \label{axiom:metric-space-03} \\
      & d(x,y)=0\implies{x=y},          \label{axiom:metric-space-04}
  \end{align}
  for all points \(x,y\) and \(z\) in \(X\). The pair \((X,d)\) is a metric
  space.
\end{definition}

We may omit the metric in \((X,d)\) and thus write `\(X\) is a metric space' if
no confusion seems possible as to which metric is being taken on \(X\). Notice
that
\[
  d(x,y)
  =
  \frac{2d(x,y)}{2}
  =
  \frac{d(x,y)+d(y,x)}{2}
  \geqslant
  \frac{d(x,x)}{2}
  =
  0,
\]
for all \(x,y\) in \(X\), so that \(d\) is actually a map from \(X\times{X}\)
into the set
\[
  \nonnegativereals
  =
  \left\{x\in\reals:x\geqslant{0}\right\},
\]
of nonnegative reals.

\subsection{Examples}

\begin{example}[The zero-one metric space]\label{example:the-0-1-metric}
  Let \(X\) be a non-empty set. Then, \((X,d)\) is a metric space, where:
  \begin{equation*}
    d(x,y)=
    \left\{
      \begin{array}{lll}
        1 & \text{if} & x=y, \\
        0 & \text{if} & x\neq{y}.
      \end{array}
    \right.
  \end{equation*}
  Although it's not a very interesting metric space, it appears frequently
  enough in counter-examples as to deserve its place in here.
\end{example}

\begin{example}[Subspaces]\label{example:subspaces}
  Let \(X\) be a metric space. Then, a non-empty subset \(Y\) of \(X\) can be
  made a metric space if one declares the distance \(d(x,y)\) between elements
  \(x,y\in{Y}\) to be the very distance from \(x\) to \(y\) as elements of
  \(X\). In this case, \(Y\) is said to be a subspace of \(X\).
\end{example}

\begin{example}[The real line]\label{example:the-real-line}
  This is the most important metric space of all.
\end{example}

\subsection{Normed linear spaces}

\[
  \norm<n>(p)[z]=\left(\sum_{j=1}^{n}\,\abs{z_{j}}^{p}\right)^{\frac{1}{p}},
\]

\begin{example}[\(p\)-norm]\label{example:p-norm}
  \(\norm(p)[u]\)
\end{example}

\subsection{Balls and spheres}

\subsection{Bounded sets}

\subsection{Distance from sets to sets}

\subsection{Isometries}

\subsection{Pseudo metrics}

\section{Exercises}

\begin{exercise}
  \TeX{}nicians will know.
\end{exercise}
