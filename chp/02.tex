\chapter{Metric Spaces}\label{chp:metric-spaces}

\section{Metric Spaces}\label{sec:metric-spaces}

\begin{definition}
  A metric on a set \(X\) is a function \(d:X\times{X}\to\reals\) such that:
  \begin{enumerate}
    \item
      \(d(x,y)=d(y,x)\),
    \item
      \(d(x,z)\leqslant{d(x,y)+d(y,z)}\),
    \item
      \(x=y\implies{d(x,y)=0}\),
    \item
      \(d(x,y)=0\implies{x=y}\),
  \end{enumerate}
  for all points \(x,y\) and \(z\) in \(X\).
\end{definition}

\[
  d(x,y)
  =
  \frac{2d(x,y)}{2}
  =
  \frac{d(x,y)+d(y,x)}{2}
  \geqslant
  \frac{d(x,x)}{2}
  =
  0,
\]

\subsection{Examples}

\subsection{Balls and spheres}

\subsection{Bounded sets}

\subsection{Distance from sets to sets}

\subsection{Isometries}

\subsection{Pseudo metrics}

\section{Exercises}

\begin{exercise}
  \TeX{}nicians will know.
\end{exercise}
